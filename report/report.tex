\documentclass[11pt]{report}
\usepackage{multicol}
\usepackage{svg}
\usepackage[french]{babel}
\usepackage[T1]{fontenc}
\usepackage[a4paper, margin=1cm, top=-1cm, bottom=-1cm]{geometry}
\usepackage[justification=centering]{caption}
\usepackage[locale=FR, group-separator={}, retain-explicit-plus]{siunitx}
\usepackage{mhchem}
\usepackage{subcaption}

\newcommand{\mychapter}[2]{
    \setcounter{chapter}{#1}
    \setcounter{section}{0}
    \chapter*{#2}
    \addcontentsline{toc}{chapter}{#2}
}

\title{
    {
        \parbox{\textwidth}{
            \textbf{\textsc{\Large Introduction aux méthodes numériques et projet}}
            \vskip 4pt \hrule height 4pt width \hsize \vskip 4pt
            \hfill \textbf{\textsc{\normalsize Année académique 2023-2024}}\\
        }
        \vfill \leavevmode
        \begin{center}
            \textbf{\Huge Production d'\ce{H2} par reformage catalytique\\ de \ce{CH4} et captage de \ce{CO2}\\ dans un réacteur.}
        \end{center}
        \includesvg[width=0.75\textwidth]{"image/3d_CO"}
        \parbox{\textwidth}{
            \includesvg[height=0.025\textheight]{"image/fsa_text"}
            \hspace{-0.4em}
            \textbf{\normalsize \hfill \today}
            \vskip 4pt \hrule height 4pt width \hsize \vskip 4pt
            \textbf{
                \normalsize
                \begin{tabular}{p{0.295\textwidth}p{0.295\textwidth}p{0.295\textwidth}}
                    \hfil \textsc{Moreci} Raffaele & \hfil \textsc{Odding} Luca & \hfil \textsc{Lindsey} Alexandre\\
                    \hfil S2304531                 & \hfil S2303933             & \hfil S2302371
                \end{tabular}
            }
        }
    }
}
\author{}
\date{}

\begin{document}    
    \maketitle

    \newgeometry{margin=2.5cm, top=2.5cm, bottom=2.5cm}
    
    \tableofcontents

    \mychapter{1}{Question 1}
        \section{Méthode de la sécante}
            Notre fonction \verb|secant| requiert trois arguments nécessaires au bon fonctionnement de la méthode qui sont :
            \begin{itemize}
                \item \verb|fun|, la fonction dont on recherche les racines
                \item \verb|x|, un tableau contenant les deux valeurs de \verb|x| de départ
                \item \verb|tol|, la tolérance qui permet de gérer la précision des résultats
            \end{itemize}
            \par
            Logiquement nous devons donc compromettre la précision pour limiter les calculs
            et donc ne pas mettre une tolérance trop faible.
            Nous avons opté, après une multitude d'essais, pour une tolérance de \num{5e-4} qui nous permet de garder 
            une bonne précision pour un temps d'exécution faible. Par exemple pour vérifier si une valeur est égale à 0,
            nous implémentons \verb|abs(y1) < tol|, les méthodes numériques ne pouvant pas être infiniment précises.
            \par
            Nous avons implémenté la fonction en s'assurant tout d'abord que toutes les conditions initiales soient bonnes
            et, pour que la fonction n'itère pas à l'infini,
            nous avons introduit \verb|max_i| en arguments \verb|kwargs| avec une valeur par défaut élevée qui restreint la fonction
            à ne pas faire plus d'itérations que ce nombre tout en laissant une marge afin de ne pas arrêter trop vite l'exécution.
            Après avoir pris nos précautions, une boucle va itérer tant que \verb|fun(x1) < tol|
            (nous utilisons de nouveau la tolérance car il est quasiment impossible que la valeur soit exactement égale à 0 par les méthodes numériques).
            \par
            Nous avons décidé de séparer le calcul du numérateur et du dénominateur afin de s'éviter certaines problématiques. Si le numérateur, \verb|y1*(x1 - x0)|,
            est égal à 0, \verb|x0| et \verb|x1| doivent avoir la même valeur. Du coup la fonction retourne \num{-1} pour indiquer que la fonction ne converge pas.
            Si le dénominateur, \verb|y1 - y0|, est égal à 0, la fonction ne converge pas vu la méthode de la sécante :
            la droite passe par deux points qui forment une droite parallèle à l'abscisse.
            La boucle se finira si il y a un non-respect de sa condition ou si la boucle fait un nombre anormalement grand d'itérations.
            
        \section{Méthode de la bissection}
            Les arguments sont les mêmes que dans la méthode de la sécante avec la même tolérance de \num{5e-4} mais sans introduire \verb|max_i|
            car si les conditions initiales sont respectées, la fonction convergera toujours.
            Nous nous assurons donc que les conditions initiales sont respectées en posant l'hypothèse que la fonctions traitée est continue.
            \par
            Pour se rapprocher de la racine nous allons itérer pendant \verb|k| itérations égal à
            $log_2(\frac{x1-x0}{2\times tol})$ comme vu dans le cours.
            Dans cette boucle, on calcule la moyenne des abscisses des deux points qui sera celle de notre nouveau point.
            Si la valeur de la fonction en ce point est positive, on remplace \verb|x1| (à image positive) par cette moyenne et inversement.
            Après toutes les itérations, la valeur de \verb|xi| sera la valeur de notre racine.

    \mychapter{2}{Question 2}
        \section{Création d'odefunction}
            Notre fonction \verb|odefunction| est une fonction qui doit recevoir trois arguments :
            \begin{itemize}
                \item \verb|z|, la distance axiale  du réacteur
                \item \verb|C0|, le tableau avec les valeurs initiales des huit variables d'état :
                    les concentrations initiales en \ce{CH4}, \ce{H2O}, \ce{H2}, \ce{CO}, \ce{CO2}
                    la conversion fractionnaire \verb|X|,
                    la température \verb|T|
                    et la pression \verb|P|
                \item \verb|mode| et \verb|param|, qui nous serviront plus tard
            \end{itemize}
            La fonction \verb|odefunction| calcule la dérivée du tableau $\frac{dC}{dz}$ et retourne ces valeurs.
            Pour plus de clarté et de flexibilité, les constantes ont été placées dans le module \verb|constants|.
            Certaines constantes sont implémentées sous forme de fonction.
            Cela n'est pas optimisé pour le temps mais nous avons opté pour cette option par soucis de lisibilité
            et de flexibilité, ce qui nous a aidé pour la question 3.

        \section{Résolution d'équation différentielle}
            Notre fonction \verb|calculConcentrationEuler| est une fonction qui suit la méthode d'Euler explicite au premier ordre
            pour résoudre approximativement l'équation contenue dans odefunction.
            Cette fonction prend cinq arguments en entrée :
            \begin{itemize}
                \item \verb|fun|, la fonction à résoudre
                \item \verb|x|, un tableau contenant les bornes pour lesquels l'équation doit être résolue
                \item \verb|y0|, les valeurs initiales de la fonction
                \item \verb|step|, le pas qui est défini par défault à \num{5e-8}
                \item \verb|mode| et \verb|param|, qui nous serviront plus tard.
            \end{itemize}
            La fonction renvoie une approximation de la fonction inconnue.
            \par
            Pour trouver le meilleur \verb|step|, nous nous sommes référés aux données de la figure \ref{tab:step}.
            Dans ce tableau crée expérimentalement, nous avons répertorié les taux d'erreur et temps d'exécution par \verb|step|.
            Pour cela, nous avons effectué plusieurs simulations sur une distance radiale réduite de \SI{0.01}{\meter} (les calculs étants plus courts).
            Pour déterminer l'erreur, nous avons fait résoudre les équations à \verb|solve_ivp| avec l'accent mis sur une grande précision. Les
            valeurs de pas qui ont un taux d'erreur acceptable sont toutes les valeurs plus petites que \num{5e-7}
            car leur taux d'erreur est en-dessous de \SI{1}{\percent} et les valeurs de pas qui prennent
            un temps acceptable sont celle plus grandes que \num{5e-9} (en prennant en compte
            que toutes ces valeurs sont sur un intervalle réduit). Du coup, nous avons choissi \num{5e-8} qui prend quasiment 2 minutes
            sur cet intervalle réduit puisque le palier suivant, \num{5e-9}, prend plus de dix fois le temps.
            Donc, grâce a toutes ces données, nous avons conclu que la valeur la plus approprié pour la
            tolérance est \num{5e-8}.
            \par
            \begin{figure}[ht]
                \centering
                \begin{tabular}{|c|c|c|}
                    \hline
                    \verb|step|        & Temps                                    & Taux d'erreur                \\
                    \hline
                    \hfill \num{5e-1}  & \hfill \SI{19.7}{\micro\second}          & \hfill \SI{61.41}{\percent}  \\
                    \hfill \num{5e-2}  & \hfill \SI{20.9}{\micro\second}          & \hfill \SI{61.41}{\percent}  \\
                    \hfill \num{5e-7}  & \hfill \SI{9.32}{\second}                & \hfill \SI{64.52}{\percent}  \\
                    \hfill \num{5e-8}  & \hfill \SI{1}{\minute} \SI{56}{\second}  & \hfill \SI{0.98}{\percent}   \\
                    \hfill \num{5e-9}  & \hfill \SI{20}{\minute} \SI{21}{\second} & \hfill \SI{0.01}{\percent}   \\
                    \hfill \num{5e-10} & \hfill \SI{3}{\hour} \SI{3}{\minute}     & \hfill \SI{0.0003}{\percent} \\
                    \hline
                \end{tabular}
                \caption{Taux d'erreur et temps d'exécution par pas décroissants}
                \label{tab:step}
            \end{figure}
            \par
            La fonction \verb|calculConcentrationIVP| prend tous les mêmes arguments que notre fonction \verb|calculConcentrationEuler|. 
            Elle utilise la fonction préexistente de \verb|scipy|, \verb|solve_IVP|.
            Nous avons modifier la valeur de la tolérance relative \verb|rtol| qui a pour but d'augmenter la précision de notre approximation.
            Nous avons choisi notre \verb|rtol| égal à \num{5e-7} puisque c'est une valeur qui apporte la précision requise en un temps d'execution imperceptible.
            \par
            Après plusieurs essais, nous pouvons conclure que \verb|solve_ivp| est beaucoup plus efficace que notre fonction
            qui prend environs 48,67 minutes pour résoudre l'équation selon le profiler intégré de spyder
            tandis que \verb|calculConcentrationIVP| ne prend que 88,69 milisecondes, soit environ 33000 fois plus rapide.
            Ceci est probablement dû au fait que \verb|solve_ivp| utilise un pas variable contrairement à notre fonction.
            La fonction \verb|solve_ivp| est également beaucoup plus précise puisqu'elle utilise une méthode d'ordre suprérieur.
            Nous concluons que \verb|calculConcentrationIVP| est beaucoup plus efficace, que ce soit pour le temps d'exécution ou pour la précision.
    \mychapter{3}{Question 3}
        Pour la question 3, nous avons introduit à \verb|odefunction| les arguments optionnels \verb|mode| et \verb|param|.
        \verb|mode| nous permet de tester différentes conditions de notre réacteur.
        Lorsque \verb|mode| est à 0, le système fonctionne avec les paramètres de base,
        lorsqu'il est à 1, le modèle ne prend pas en compote la carbonatation,
        lorsqu'il est à 2, on remplace \verb|us| par la valeur de \verb|param|,
        lorsqu'il est à 3, on remplace \verb|ug| par la valeur de \verb|param|
        et lorsqu'il est a 4, on remplace \verb|us| par \verb|param[0]| et \verb|TW| par \verb|param[1]|.
        \section{Modèle sans capture de \ce{CO2}}
            \begin{figure}[ht]
                \centering
                \begin{minipage}{0.45\textwidth}
                    \centering
                    \includesvg[width=0.95\textwidth]{"image/con_mode0"}
                    \caption{Variation des\\ concentrations avec carbonatation}
                    \label{graph:con:0}
                \end{minipage}
                \begin{minipage}{0.45\textwidth}
                    \centering
                    \includesvg[width=0.95\textwidth]{"image/con_mode1"}
                    \caption{Variation des\\ concentrations sans carbonatation}
                    \label{graph:con:1}
                \end{minipage}
            \end{figure}
            Afin de comparer un modèle de réacteur avec et sans capture de \ce{CO2},
            nous avons mis le mode de odefunction à 1, soit un modèle sans capture de \ce{CO2}.
            Pour simuler ceci, nous avons remplacé la conversion fractionnaire (\verb|X|),
            le taux de consommation de \ce{CO2} par carbonatation (\verb|rcbn|)
            et la vitesse d'entré du \ce{CaO} dans le réacteur (\verb|us|)
            par 0 car ils n'interviennent plus dans le modèle sans capture de \ce{CO2}.
            \begin{figure}[ht]
                \centering
                \begin{tabular}{|c|c|c|c|}
                    \hline
                    Nom                  & \verb|mode| 0                       & \verb|mode| 1                       & Différence                  \\
                    \hline
                    \ce{CH4}             & \hfill \SI{0.00417}{\mol\per\liter} & \hfill \SI{0.00478}{\mol\per\liter} & \hfill \SI{+14.7}{\percent} \\
                    \ce{H2}              & \hfill \SI{0.02322}{\mol\per\liter} & \hfill \SI{0.02302}{\mol\per\liter} & \hfill \SI{-0.9}{\percent}  \\
                    \ce{CO}              & \hfill \SI{0.00235}{\mol\per\liter} & \hfill \SI{0.00250}{\mol\per\liter} & \hfill \SI{+6.5}{\percent}  \\
                    \ce{CO2}             & \hfill \SI{0.00344}{\mol\per\liter} & \hfill \SI{0.00387}{\mol\per\liter} & \hfill \SI{+12.6}{\percent} \\
                    \verb|T|\textdegree  & \hfill \SI{920.23}{\kelvin}         & \SI{915.85}{\kelvin}                & \hfill \SI{-0.5}{\percent}  \\
                    \hline
                \end{tabular}
                \caption{Effet de la carbonatation sur le modèle}
                \label{tab:carb}
            \end{figure}
            \par
            Pour comparer un système avec et sans carbonatation,
            il est difficile d'utiliser ces graphiques car,
            comme on peut le voir aux figures \ref{graph:con:0} et \ref{graph:con:1},
            la tendance générale du graphique ne varie pas énormément avec ou sans carbonatation mais ils peuvent être utiles 
            pour distinguer les différences se créer au fil de l'avancée dans le réacteur.
            C'est pour cela que nous allons nous fier aux valeurs finales des concentrations (figure \ref{tab:carb}).
            Donc, dans un modèle sans carbonatation, on remarque que le \ce{CH4}, le \ce{CO2}
            et le \ce{CO} augmentent de \SI{14.7}{\percent}, \SI{12.6}{\percent} et \SI{6.5}{\percent} respectivement.
            Le \ce{H2} et la température (\verb|T|) diminuent de \SI{0.9}{\percent} et \SI{0.5}{\percent} respectivement.
        \section{Effet de l'augmentation du flux des solides sur le modèle}
            Pour analyser l'effet du flux des solides,
            nous avons fait varier la valeur de \verb|us| à l'aide du \verb|mode| 2.
            Nous l'avons fait varier de \SI{0.001}{\meter\per\second} à \SI{0.025}{\meter\per\second}
             et créer ces graphiques pour chaque variable d'état indépendamment avec plusieurs valeurs de \verb|us|.
            \begin{figure}[ht]
                \centering
                \begin{subfigure}[t]{0.325\textwidth}
                    \centering
                    \includesvg[width=0.95\textwidth]{"image/CH4_mode2"}
                    \caption{\ce{CH4}}
                    \label{graph:con:2:CH4}
                \end{subfigure}
                \hfill
                \begin{subfigure}[t]{0.325\textwidth}
                    \centering
                    \includesvg[width=0.95\textwidth]{"image/H2_mode2"}
                    \caption{\ce{H2}}
                    \label{graph:con:2:H2}
                \end{subfigure}
                \hfill
                \begin{subfigure}[t]{0.325\textwidth}
                    \centering
                    \includesvg[width=0.95\textwidth]{"image/CO_mode2"}
                    \caption{\ce{CO}}
                    \label{graph:con:2:CO}
                \end{subfigure}
                \hfill
                \begin{subfigure}[t]{0.325\textwidth}
                    \centering
                    \includesvg[width=0.95\textwidth]{"image/CO2_mode2"}
                    \caption{\ce{CO2}}
                    \label{graph:con:2:CO2}
                \end{subfigure}
                \begin{subfigure}[t]{0.325\textwidth}
                    \centering
                    \includesvg[width=0.95\textwidth]{"image/T_mode2"}
                    \caption{Temperature (\texttt{T})}
                    \label{graph:con:2:T}
                \end{subfigure}
                \caption{Concentrations et températures (\texttt{T}) des gaz}
                \label{graph:con:2}
            \end{figure}
            De manière générale, les variations de  vont impacter relativement beaucoup 
            les concentrations et la température au début du réacteur mais à la sortie,
            l'écart avec le us de base (\SI{0.001}{\meter\per\second}), est plus faible. C'est pour cela que 
            nous n'avons pas pris une plage de valeur de us plus grande, nous voyons avec 
            ces valeurs une bonne représentation de ce qui se passe avec des valeurs de 
            \verb|us| > 0.001. Nous pouvons séparer toutes les variables d'états en deux comportements 
            différents : celles qui diminuent au fur et à mesure que la valeur de us augmente
            qui sont le \ce{CH4} et le \ce{CO2} et celle qui augmentent au fur et mesure que us augmente, 
            à savoir le \ce{H2}, le \ce{CO}, et la température.
            Afin de mieux comprendre quels sont les variations des gaz à la sortie du réacteur, 
            pour mieux voir les quantités que l'on récupère,
            voici un autre graphique des concentrations des gaz secs en fonction de \verb|us|.
            (((GRAPHIQUEEE)))
        
        \section{Effets de l'augmentation du flux des gaz sur le modèle}
            Pour analyser le flux des gaz, nous avons fait varier \verb|ug| à l'aide du \verb|mode| 3
            en changeant la valeur de \SI{1}{\meter\per\second} à \SI{3}{\meter\per\second} avec un pas de 0.5.
            (((EXPLIQUER GRAPHIQUES)))
            \begin{figure}[ht]
                \centering
                \begin{subfigure}[t]{0.325\textwidth}
                    \centering
                    \includesvg[width=0.95\textwidth]{"image/CH4_mode3"}
                    \caption{\ce{CH4}}
                    \label{graph:con:3:CH4}
                \end{subfigure}
                \hfill
                \begin{subfigure}[t]{0.325\textwidth}
                    \centering
                    \includesvg[width=0.95\textwidth]{"image/H2_mode3"}
                    \caption{\ce{H2}}
                    \label{graph:con:3:H2}
                \end{subfigure}
                \hfill
                \begin{subfigure}[t]{0.325\textwidth}
                    \centering
                    \includesvg[width=0.95\textwidth]{"image/CO_mode3"}
                    \caption{\ce{CO}}
                    \label{graph:con:3:CO}
                \end{subfigure}
                \hfill
                \begin{subfigure}[t]{0.325\textwidth}
                    \centering
                    \includesvg[width=0.95\textwidth]{"image/CO2_mode3"}
                    \caption{\ce{CO2}}
                    \label{graph:con:3:CO2}
                \end{subfigure}
                \begin{subfigure}[t]{0.325\textwidth}
                    \centering
                    \includesvg[width=0.95\textwidth]{"image/T_mode3"}
                    \caption{Temperature (\texttt{T})}
                    \label{graph:con:3:T}
                \end{subfigure}
                \caption{Concentrations et températures (\texttt{T}) des gaz}
                \label{graph:con:3}
            \end{figure}
        
        \section{Effet de la composition du gaz d'entrée sur le modèle}
            Dans le but de modifier les compositions initiales de gaz,
            nous avons fait varier le rapport des concentrations initiales en \ce{H2O} et \ce{CH4} de 3 à 0.3.

            Comme illustré sur les 5 graphiques ci-dessous, l'augmentation de ug produit exactement l'effet 
            inverse à l'augmentation de \verb|us|. Au début, les données sont très sensibles aux variation de \verb|ug| 
            et très peu sensible a ces variation à la fin du réacteur. Nous pouvons aussi séparer les données 
            de sorties en deux, mais cette fois-ci ce sont le \ce{H2}, le \ce{CO} et la température qui diminuent 
            tandis que le \ce{CH4} et le \ce{CO2} augmentent quand la valeur de \verb|ug| augmente
            \begin{figure}[ht]
                \centering
                \begin{subfigure}[t]{0.325\textwidth}
                    \centering
                    \includesvg[width=0.95\textwidth]{"image/CH4_C0"}
                    \caption{\ce{CH4}}
                    \label{graph:con:C0:CH4}
                \end{subfigure}
                \hfill
                \begin{subfigure}[t]{0.325\textwidth}
                    \centering
                    \includesvg[width=0.95\textwidth]{"image/H2_C0"}
                    \caption{\ce{H2}}
                    \label{graph:con:C0:H2}
                \end{subfigure}
                \hfill
                \begin{subfigure}[t]{0.325\textwidth}
                    \centering
                    \includesvg[width=0.95\textwidth]{"image/CO_C0"}
                    \caption{\ce{CO}}
                    \label{graph:con:C0:CO}
                \end{subfigure}
                \hfill
                \begin{subfigure}[t]{0.325\textwidth}
                    \centering
                    \includesvg[width=0.95\textwidth]{"image/CO2_C0"}
                    \caption{\ce{CO2}}
                    \label{graph:con:C0:CO2}
                \end{subfigure}
                \begin{subfigure}[t]{0.325\textwidth}
                    \centering
                    \includesvg[width=0.95\textwidth]{"image/T_C0"}
                    \caption{Temperature (\texttt{T})}
                    \label{graph:con:C0:T}
                \end{subfigure}
                \caption{Concentrations et températures (\texttt{T}) des gaz}
                \label{graph:con:C0}
            \end{figure}

    \mychapter{4}{Question 4}
        \section{Taux de \ce{CO2} optimal par rapport à \texttt{ug}}
            \verb|optimise_us| crée la fonction du pourcentage de \ce{CO2} en sortie par rapport à \verb|us|,
            ce qui nous permet de déterminer un \verb|us| optimal pour une concentration donnée.
            Cette fonction prend 5 arguments en entrée :
            \begin{itemize}
                \item \verb|Y|, le pourcentage de \ce{CO2} souhaité à la sortie du réacteur
                \item \verb|us|, la valeur de \verb|us|
                \item \verb|C0|, le tableau des conditions initiales pour \verb|odefunction|
                \item \verb|mode_| et \verb|param1|, équivalents du \verb|mode| et \verb|param| de \verb|odefunction|
            \end{itemize}
            Cette fonction résout odefunction grâce à \verb|solve_ivp| et retourne la concentration
            finale de \ce{CO2} divisé par la somme des concentration finale de \ce{CH4}, \ce{H2},
            \ce{CO} et de \ce{CO2}. Le tout est ensuite soustrait par $Y$ pour corriger le décalage vertical.
        \section{Recherche d'un \texttt{us} optimal}
            \begin{figure}[ht]
                \centering
                \includesvg[width=0.5\textwidth]{"image/opt_us"}
                \caption{Valeur optimale de \texttt{us}}
                \label{graph:us}
            \end{figure}
            Dans ce cas, nous devons trouvez la valeur optimale de \verb|us| pour une concentration
            en \ce{CO2} de \SI{7.5}{\percent}.
            On peut voir la fonction \verb|optimise_us| ainsi que la valeur optimale sur la figure \ref{graph:us}
            qui représente \verb|us| en fonction du pourcentage de \ce{CO2} parmis les gaz secs en sortie.
            Notre fonction \verb|secant| en mode \verb|hybrid| nous renvoit une racine située en 0.103.
            \par
            La méthode utilisée pour trouver la racine est une méthode hybride qui est simplement
            la méthode de la sécante légèrement modifiée : si la sécante ne converge pas dû à des
            images de \verb|x0| et \verb|x1| égales,
            alors on s'inspire de la méthode de la bissection pour  débloquer la situation.
            Pour ce faire, on effectue la moyenne des deux points et on utilise ce nouveau point pour relancer
            la fonction \verb|secant|. Ce faisant, on bénéficie de la précision accrue de la méthode de la sécante
            ainsi que sa vitesse bien supérieure a la bissection. De plus, la méthode de la sécante ne demande 
            pas d'avoir \verb|fun(x0)| et \verb|fun(x1)| de signes contraires. C'est pour ces raisons que \verb|secant| nous a 
            paru être la bonne fonction a utiliser.
        \section{Effet de la température sur le \texttt{us} optimal}
            La figure \ref{graph:us/T} illustre la variation du \verb|us| optimal en fonction de la température (\verb|T|).
            Il est clair que \verb|us| dépend de la température. 
            Nous le voyons décroitre très rapidement, presque exponentiellement à mesure que la température augmente.
            Donc, plus la température est élevée, plus la vitesse d'entrée
            optimale des gaz pour un pourcentage de \ce{CO2} donné est petite.
            \begin{figure}[ht]
                \centering
                \includesvg[width=0.5\textwidth]{"image/opt_us-T"}
                \caption{L'effet de la température (\texttt{T}) sur le \texttt{us} optimal}
                \label{graph:us/T}
            \end{figure}

\end{document}